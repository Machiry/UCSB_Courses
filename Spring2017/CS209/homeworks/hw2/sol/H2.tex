\documentclass[12pt,letterpaper]{article}
\usepackage{times}
\usepackage{latexsym}
\usepackage{amssymb}
\usepackage[bookmarks=false,pdfstartview=FitH]{hyperref}

\topmargin -1.25 in
\oddsidemargin 0 in
\evensidemargin 0 in
\textheight 10 in
\textwidth 6.5 in
\parindent 5mm
\leftmargini 5mm
\leftmarginii 5mm
\def\le{\leqslant}
\def\preceq{\preccurlyeq}
\def\faA{\mathfrak{A}}
\def\faB{\mathfrak{B}}
\def\faC{\mathfrak{C}}
\def\caR{{\cal R}}
\def\Emodels{\models\hspace*{-3pt}{=}\hspace*{-6pt}\mid}

\title{\bf Homework \#2\vspace*{-5mm} (CS209, Spring 2017)}
\author{\normalsize Aravind Machiry}
\date{\vspace*{-1cm}}

\begin{document}
\maketitle
\thispagestyle{empty}
\begin{enumerate}
\item
  Consider a (first-order) logic language with the following parameters:
  $\forall$,
  $Px$ ($x$ is a person),
  $Ty$ ($y$ is a time instant),
  $Fxy$ (you can fool person $x$ at time instant $y$),
  Translate the following English sentences
  into this logic language.
  (If a sentence is ambiguous and you will need
  more than one translation.)
  \vspace*{-2mm}
  \begin{enumerate}\itemsep 0pt
  \item You can fool some of the people all of the time.

\textbf{Answer:}
$\exists x\ Px \land (\forall y\ (Ty \rightarrow Fxy))$ which is same as $\neg \forall x\ \neg (Px \land (\forall y\ (Ty \rightarrow Fxy)))$
  
  \item You can fool all of the people some of the time.

\textbf{Answer:}
$\exists y\ Ty \land (\forall x\ (Px \rightarrow Fxy))$ which is same as $\neg \forall y\ \neg (Ty \land (\forall x\ (Px \rightarrow Fxy)))$
  
  \item You can't fool all of the people all of the time.
  
  \textbf{Answer:}
  $\neg (\forall x \forall y\ (Px \land Ty \land Fxy))$
  \end{enumerate}

\item
  Let $\Gamma$ be a set of wffs,
  $\varphi$ and $\psi$ be wffs
  in some first-order language $\cal L$.
  Prove the following:
  \vspace*{-2mm}
  \begin{enumerate}\itemsep 0pt
  \item
    $\Gamma;\varphi\models\psi$ ~iff~
    $\Gamma\models(\varphi\rightarrow\psi)$.
  \item
    $\varphi\Emodels\psi$ ~iff~
    $\models(\varphi\leftrightarrow\psi)$.
  \end{enumerate}

\item
  Let $\alpha$ and $\beta$ be two wffs.
  Prove the following:~
  $\{\forall x(\alpha\rightarrow\beta), \forall x \alpha\}
  \models \forall x \beta$.

\item
  Show that a wff $\theta$ is valid iff $\forall x \theta$ is valid.

\item
  Let $\cal L$ be a logic language with one unary predicate
  symbol $R$, no contant symbols, no function symbols.
  Let $V$ be a set of variables.
  We define the set of wffs
  $\Gamma=\{\neg Ry\mid y\in V\}\cup\{\exists x\,Rx\}$.
  Is $\Gamma$ satisfiable? Prove your answer.

\item
  Let $\mathfrak{R}$ be
  the structure $({\mathbb R}, +, \times)$ for real numbers
  for the language consists of $\forall, +$, and $\times$
  (no constants).
  The addition and multiplication operations in $\mathfrak{R}$
  are the usual operations.
  \vspace*{-2mm}
  \begin{enumerate}\itemsep 0pt
  \item
    Write a formula that defines the set $\{0\}$
    (i.e., containing a single element $0$).
  \item
    Write a formula that defines the interval $[0,\infty)$
    (a set of real numbers).
  \item
    Write a formula that defines the set $\{2\}$.
  \end{enumerate}

\item
  Consider a new quantifier: $\exists!x\alpha$
  (read ``there exists a unique $x$ such that $\alpha$'')
  is to be satisfied in a structure $\faA$ with an assignment $v$
  iff there is one and only one element $d\in|\faA|$ such that
  $\models_{\faA}\alpha[v(x/d)]$.
  Assume that the language has the equality symbol.
  Find a formula in the original language that
  is logically equivalent to $\exists!x\alpha$.

\item
  Prove Part (3) and Part (4) of the Homomorphism Theorem.

\item
  Let $h$ be an isomorphic embedding of $\faA$ into $\faB$.
  Show that there is a structure $\faC$ isomorphic to $\faB$
  such that $\faA$ is a substructure of $\faC$.\\
  Hint:
  Let $g$ be a one-to-one function with domain $|\faB|$  such that
  $g(h(a))=a$ for all $a\in|\faA|$.
  Define the structure $\faC$ such that
  $g$ is an isomorphism onto $\faC$.

\item
  Consider the structure structure $({\mathbb R}, +)$
  of real numbers for the language
  consisting of $\forall, +$ (no multiplication
  nor constants).
  The addition operation is the usual operations.
  Prove that the set $\{1\}$ cannot be defined by any formula.
  \\
  Hint: Consider automorphisms defined by linear functions.
\end{enumerate}

\end{document}
