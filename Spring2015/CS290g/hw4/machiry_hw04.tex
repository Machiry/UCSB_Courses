\documentclass[11pt, pdftex]{article}
\usepackage[margin=1in]{geometry}
\usepackage{graphicx}
\usepackage{amsmath}
\usepackage{listings}
\usepackage{semantic}
\usepackage[hyphens]{url}
\usepackage[breaklinks]{hyperref}
\usepackage[demo]{graphicx}
\usepackage{subcaption}
\title{Homework Assignment 04}
\author{Machiry Aravind Kumar}
\date{UCSB}
\begin{document}
\maketitle
\section{ Implement the Fermat’s primality testing algorithm in Python, C++ or Mathematica, and apply to these numbers and discover the smallest liar and smallest witness for each one. What is the property of the witness?}
I implemented the algorithm in python. File $\texttt{machiry\_hw4.py}$ has the code for the algorithm. I used Montgomery exponentiation to perform $a^{p-1} mod n$ check. The results for smallest witness and smallest liar for the given numbers are as shown in Table \ref{tab:fermat}.
\begin{center}
\begin{tabular}{ |c|c|c| }
\hline
Target Number & Smallest Witness & Smallest Liar\\
\hline
41041 & 7 & 2\\
\hline
62745 & 3 & 2\\
\hline
63973 & 7 & 2\\
\hline
75361 & 11 & 2\\
\hline
101101 & 7 & 2\\
\hline
126217 & 7 & 2\\
\hline
172081 & 7 & 2\\
\hline
188461 & 7 & 2\\
\hline
278545 & 5 & 2\\
\hline
340561 & 13 & 2\\
\hline
449065 & 5 & 2\\
\hline
552721 & 13 & 2\\
\hline
656601 & 3 & 2\\
\hline
658801 & 11 & 2\\
\hline
670033 & 7 & 2\\
\hline
748657 & 7 & 2\\
\hline
838201 & 7 & 2\\
\hline
852841 & 11 & 2\\
\hline
997633 & 7 & 2\\
\hline
1033669 & 7 & 2\\
\hline
1082809 & 7 & 2\\
\hline
1569457 & 17 & 2\\
\hline
1773289 & 7 & 2\\
\hline
2100901 & 11 & 2\\
\hline
2113921 & 19 & 2\\
\hline
2433601 & 17 & 2\\
\hline
2455921 & 13 & 2 \\
\hline
\end{tabular}
\label{tab:fermat}
\end{center}
The property of smallest witness of all these numbers are they are the \textbf{smallest prime factors of corresponding carmichael numbers}.
\section{Implement the Miller-Rabin primality testing algorithm in Python, C++ or Mathematica, and apply to these numbers and discover the smallest liar and smallest witness for each one.}
I implemented the algorithm in python. File $\texttt{machiry\_hw4.py}$ has the code for the algorithm. I used Montgomery exponentiation to perform all modular exponentiations. The results for smallest witness and smallest liar for the given numbers are as shown in Table \ref{tab:miller}.
\begin{center}
\begin{tabular}{ |c|c|c| }
\hline
Target Number & Smallest Witness & Smallest Liar\\
\hline
41041 & 2 & 16\\
\hline
62745 & 2 & 16\\
\hline
63973 & 2 & 9\\
\hline
75361 & 2 & 256\\
\hline
101101 & 2 & 16\\
\hline
126217 & 2 & 16\\
\hline
172081 & 2 & 9\\
\hline
188461 & 2 & 9\\
\hline
278545 & 2 & 98\\
\hline
340561 & 2 & 35\\
\hline
449065 & 2 & 16\\
\hline
552721 & 2 & 21\\
\hline
656601 & 2 & 16\\
\hline
658801 & 2 & 101\\
\hline
670033 & 2 & 9\\
\hline
748657 & 2 & 9\\
\hline
838201 & 2 & 9\\
\hline
852841 & 2 & 16\\
\hline
997633 & 2 & 898\\
\hline
1033669 & 2 & 9\\
\hline
1082809 & 2 & 16\\
\hline
1569457 & 2 & 256\\
\hline
1773289 & 2 & 3\\
\hline
2100901 & 2 & 16\\
\hline
2113921 & 2 & 195\\
\hline
2433601 & 2 & 98\\
\hline
2455921 & 2 & 9\\
\hline
\end{tabular}
\label{tab:miller}
\end{center}
\end{document}