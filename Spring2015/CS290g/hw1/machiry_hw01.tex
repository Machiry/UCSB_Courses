\documentclass[11pt, pdftex]{article}
\usepackage[margin=1in]{geometry}
\usepackage{graphicx}
\usepackage{amsmath}
\usepackage{listings}
\usepackage{semantic}
\usepackage[hyphens]{url}
\usepackage[breaklinks]{hyperref}
\usepackage[demo]{graphicx}
\usepackage{subcaption}
\title{Homework Assignment 01}
\author{Machiry Aravind Kumar}
\date{UCSB}
\begin{document}
\maketitle
\section{Problem 1}
Let e = (10001111) be the exponent. Illustrate the addition chains produced by each one of the following algorithms. Compute the length of each addition chain.
\subsection{Binary method}
\begin{center}
\begin{tabular}{ |c|c|l|l| } 
 \hline
 i & e_{i} & Step 2a & Step 2b \\
 \hline
 \hline 
 6 & 0 & (M)^{2} =  M^{2} & M^{2} \\ 
 5 & 0 & (M^{2})^{2} = M^{4} & M^{4} \\ 
 4 & 0 & (M^{4})^{2} = M^{8} & M^{8} \\
 3 & 1 & (M^{8})^{2} = M^{16} & M^{16}.M = M^{17} \\ 
 2 & 1 & (M^{17})^{2} = M^{34} & M^{34}.M = M^{35} \\ 
 1 & 1 & (M^{35})^{2} = M^{70} & M^{70}.M = M^{71} \\ 
 0 & 1 & (M^{71})^{2} = M^{142} & M^{142}.M = M^{143} \\
 \hline
\end{tabular}
\end{center}
Addition chain = 1 2 4 8 16 17 34 35 70 71 142 143, Length = 12.
\subsection{m-ary method}
\subsubsection{For d=2}
\begin{center}
\begin{tabular}{ |c|c|l|} 
 \hline
 bits & w & M^{w} \\
 \hline
 \hline 
 00 & 0 & 1 \\ 
 01 & 1 & M \\ 
 10 & 2 & M.M = M^{2} \\
 11 & 3 & M^{2}.M = M^{3} \\
 \hline
\end{tabular}
\end{center}

\begin{center}
\begin{tabular}{ |c|c|l|l| } 
 \hline
 i & F_{i} & Step 4a & Step 4b \\
 \hline
 \hline 
 2 & 00 & (M^{2})^{4} = M^{8} & M^{8} \\ 
 1 & 11 & (M^{8})^{4} = M^{32} & M^{32}.M^{3} = M^{35} \\ 
 0 & 11 & (M^{35})^{4} = M^{140} & M^{140}.M^{3} = M^{143} \\
 \hline
\end{tabular}
\end{center}
Addition chain = 1 2 3 4 8 16 32 35 70 140 143, Length = 11.
\subsubsection{For d=4}
\begin{center}
\begin{tabular}{ |c|c|l|} 
 \hline
 bits & w & M^{w} \\
 \hline
 \hline
 0000 & 0 & 1 \\ 
 0001 & 1 & M \\ 
 0010 & 2 & M.M = M^{2} \\
 0011 & 3 & M^{2}.M = M^{3} \\
 0100 & 4 & M^{3}.M = M^{4} \\ 
 0101 & 5 & M^{4}.M = M^{5} \\ 
 0110 & 6 & M^{5}.M = M^{6} \\
 0111 & 7 & M^{6}.M = M^{7} \\
 1000 & 8 & M^{7}.M = M^{8} \\ 
 1001 & 9 & M^{8}.M = M^{9} \\ 
 1010 & 10 & M^{9}.M = M^{10} \\
 1011 & 11 & M^{10}.M = M^{11} \\
 1100 & 12 & M^{11}.M = M^{12} \\ 
 1101 & 13 & M^{12}.M = M^{13} \\ 
 1110 & 14 & M^{13}.M = M^{14} \\
 1111 & 15 & M^{14}.M = M^{15} \\
 \hline
\end{tabular}
\end{center}

\begin{center}
\begin{tabular}{ |c|c|l|l| } 
 \hline
 i & F_{i} & Step 4a & Step 4b \\
 \hline
 \hline 
 0 & 1111 & (M^{8})^{16} = M^{128} & M^{128}.M^{15} = M^{143} \\
 \hline
\end{tabular}
\end{center}
Addition chain = 1 2 3 4 5 6 7 8 9 10 11 12 13 14 15 16 32 64 128 143, Length = 20.
\subsection{Factor Method}
Compute: M $ \rightarrow$ M^{2} \\
Assign: a = M^{2} \\
Compute: a $ \rightarrow$ a^{2} $ \rightarrow$ a^{4} $ \rightarrow$ a^{5}\\
Assign: b = a^{5} = M^{10}\\
Compute: b.M $ \rightarrow$ M^{11} \\
Assign: c = M^{11} \\
Compute: c $ \rightarrow$ c^{2} $ \rightarrow$ c^{3} \\
Assign: d = c^{3} = (M^{11})^{3}\\
Compute: d $ \rightarrow$ d^{2} $ \rightarrow$ d^{4} \\
Assign: e = d^{4} = (M^{11})^{12} \\
Compute: e.c $ \rightarrow$  (M^{11})^{13} = $M^{143}$ \\
Addition chain = 1 2 4 8 10 11 22 33 66 132 143, Length = 11.
\subsection{Power Tree Method}
A tree of height 11 leads to 143 as its leaf node. Refer code $\texttt{mk\_tree.py}$ for the details. 
The path from the root is: 1 2 3 5 7 14 21 35 70 140 143. Addition chain is same as this path.\\
Addition chain = 1 2 3 5 7 14 21 35 70 140 143, Length = 11.
\subsection{Canonical Recording for d=1}
Truth table for canonical recording is as shown below:
\begin{center}
\begin{tabular}{ |c|c|c|c|c|} 
 \hline
 c_{i} & e_{i+1} & e_{i} & c_{i+1} & f_{i} \\
 \hline
 \hline 
 0 & 0 & 0 & 0 & 0 \\ 
 0 & 0 & 1 & 0 & 1 \\ 
 0 & 1 & 0 & 0 & 0 \\
 0 & 1 & 1 & 1 & $\overline{1}$\\
 1 & 0 & 0 & 0 & 1 \\ 
 1 & 0 & 1 & 1 & 0 \\ 
 1 & 1 & 0 & 1 & $\overline{1}$\\
 1 & 1 & 1 & 1 & 0 \\
 \hline
\end{tabular}
\end{center}
Using truth table, recording for given number is:
\begin{center}
\begin{tabular}{ |c|c|c|c|c|} 
 \hline
 c_{i} & e_{i+1} & e_{i} & c_{i+1} & f_{i} \\
 \hline
 \hline 
 0 & 1 & 1 & 1 & $\overline{1}$ \\ 
 1 & 1 & 1 & 1 & 0 \\ 
 1 & 1 & 1 & 1 & 0 \\
 1 & 0 & 1 & 1 & 0\\
 1 & 0 & 0 & 0 & 1 \\ 
 0 & 0 & 0 & 0 & 0 \\ 
 0 & 1 & 0 & 0 & 0\\
 0 & 0 & 1 & 0 & 1\\
 \hline
\end{tabular}
\end{center}
$1001000\overline{1}$ = 2^{7} + 2^{4} - 2^{0} = 143\\

Computing the exponent:
\begin{center}
\begin{tabular}{ |c|c|l|l| } 
 \hline
 i & f_{i} & Step 2a & Step 2b \\
 \hline
 \hline 
 7 & 1 & M & M \\ 
 6 & 0 & (M)^{2} =  M^{2} & M^{2} \\ 
 5 & 0 & (M^{2})^{2} = M^{4} & M^{4} \\ 
 4 & 1 & (M^{4})^{2} = M^{8} & M^{8}.M=M^{9} \\
 3 & 0 & (M^{9})^{2} = M^{18} & M^{18} \\ 
 2 & 0 & (M^{18})^{2} = M^{36} & M^{36} \\ 
 1 & 0 & (M^{36})^{2} = M^{72} & M^{72} \\ 
 0 & $\overline{1}$ & (M^{72})^{2} = M^{144} & M^{144}.$M^{-1}$ = M^{143} \\
 \hline
\end{tabular}
\end{center}
Addition chain: 1 2 4 8 9 18 36 72 144 143, Length = 10.
\section{Illustrate the steps of the standard multiplication algorithm for computing c
=a * b = 456 * 555}
\begin{center}
\begin{tabular}{ ccccc} 
 \hline
 i & j & Step & (C,S) & Partial t \\
 \hline
 0 & 0 & t_{0}+a_{0}b_{0}+C & (0,*) & 000000 \\ 
  &  & 0 + 6*5 + 0 & (3,0) & 00000$\textbf{0}$ \\ 
\hline
  & 1 & t_{1}+a_{1}b_{0}+C &  &  \\ 
  &  & 0 + 5*5 + 3 & (2,8) & 0000$\textbf{8}$0 \\ 
 \hline
  & 2 & t_{2}+a_{2}b_{0}+C &  &  \\ 
  &  & 0 + 4*5 + 2 & (2,2) & 000$\textbf{2}$80 \\ 
 \hline
 & & & & 00$\textbf{2}$280 \\
 \hline
  1 & 0 & t_{1}+a_{0}b_{1}+C & (0,*) &  \\ 
  &  & 8 + 6*5 + 0 & (3,8) & 0022$\textbf{8}$0 \\ 
\hline
  & 1 & t_{2}+a_{1}b_{1}+C &  &  \\ 
  &  & 2 + 5*5 + 3 & (3,0) & 002$\textbf{0}$80 \\ 
 \hline
  & 2 & t_{3}+a_{2}b_{1}+C &  &  \\ 
  &  & 2 + 4*5 + 3 & (2,5) & 00$\textbf{5}$080 \\ 
 \hline
 & & & & 0$\textbf{2}$5080 \\
 \hline
  2 & 0 & t_{2}+a_{0}b_{2}+C & (0,*) &  \\ 
  &  & 0 + 6*5 + 0 & (3,0) & 025$\textbf{0}$80 \\ 
\hline
  & 1 & t_{3}+a_{1}b_{2}+C &  &  \\ 
  &  & 5 + 5*5 + 3 & (3,3) & 02$\textbf{3}$080 \\ 
 \hline
  & 2 & t_{4}+a_{2}b_{2}+C &  &  \\ 
  &  & 2 + 4*5 + 3 & (2,5) & 0$\textbf{5}$3080 \\ 
 \hline
 & & & & $\textbf{2}$53080 \\
 \hline
\end{tabular}
\end{center}
\section{Illustrate the steps of the standard squaring algorithm for computing c
=a * a = 456 * 456}
\begin{center}
\begin{tabular}{ ccccc} 
 \hline
 i & j & Step & (C,S) & Partial t \\
 \hline
 0 & 1 & t_{0} + a_{0}a_{0} &  & 000000 \\
  & & 0 + 6*6 & (3,6) & 00000$\textbf{6}$ \\
  & & t_{1}+2a_{1}a_{0}+C & (3,*) & 000006 \\ 
  & & 0 + 2*5*6 + 3 & (6,3) & 0000$\textbf{3}$6 \\ 
\hline
 0 & 2 & t_{2}+2a_{2}a_{0}+C & (6,*) & 000036 \\ 
  &  & 0 + 2*4*6 + 6 & (5,4) & 000$\textbf{4}$36 \\ 
 \hline
 & & & & 00$\textbf{5}$436 \\
 \hline
  1 & 2 & t_{2} + a_{1}a_{1} &  & 005436 \\
  & & 4 + 5*5 & (2,9) & 005$\textbf{9}$36 \\
  & & t_{3}+2a_{2}a_{1}+ C & (2,*) & 005936 \\ 
  & & 5 + 2*4*5 + 2 & (4,7) & 00$\textbf{7}$936 \\ 
\hline
 & & & & 0$\textbf{4}$7936 \\
 \hline
 2 & 2 & t_{4} + a_{2}a_{2} &  & 047936 \\
  & & 4 + 4*4 & (2,0) & 0$\textbf{0}$7936 \\
 \hline
 & & & & $\textbf{2}$07936 \\
 \hline
\end{tabular}
\end{center}
\section{Let r = 32, n = 21, a = 13, and b = 15. Compute $c = a * b * r^{-1}$ mod n using the standard Montgomery multiplication algorithm. Illustrate the steps and give all temporary results}
\begin{center}
\begin{tabular}{c||c|c|c|c|c|c|c} 
iteration & q & g_{0} & g_{1} & u_{0} & u_{1} & v_{0} & v_{1}\\
\hline
0 & - & 32 & 21 & 1 & 0 & 0 & 1 \\
\hline
1 & 1 & 21 & 11 & 0 & 1 & 1 & -1 \\
\hline
2 & 1 & 11 & 10 & 1 & -1 & -1 & 2\\
\hline
3 & 1 & 10 & 1 & -1 & 2 & 2 & -3\\
\hline
4 & 10 & 1 & 0 & 2 & -21 & -3 & 32 \\
\hline
& & $\bullet$ & & $\bullet$ & & $\bullet$ & \\
\label{tab:gcd1}
\end{tabular}
\end{center}
From Table \ref{tab:gcd1}, GCD = 1, $r^{-1}$ = 2, $n'$ = 3. \\
Consider $\overline{x}$ = a = 13, such that x = $\overline{x} * r^{-1} $ mod n = 5. $\overline{y}$ = b = 15, such that y = $\overline{y} * r^{-1} $ mod n = 9. \\
Now, $a * b * r^{-1}$ mod n is same as $\overline{x} * \overline{y} * r^{-1}$ mod n.\\
So, $a * b * r^{-1}$ mod n = $\overline{x} * \overline{y} * r^{-1}$ mod n = MonPro($\overline{x}$ = 13,$\overline{y}$ = 15).

\subsection{function MonPro($\overline{a}$ = 13 , $\overline{b}$ = 15)}
\begin{enumerate}
\item t = 13*15 = 195
\item m = (195*3) mod 32 = 9
\item u = (195 + 9 * 21) / 32 = 384/32 = 12
\item 12 $<$ 21 $\textbf{return}$ 12
\end{enumerate}
Final result is : 12.
\section{Let p = 29, a = 23, and g = 10. Compute $g^{a}$ (mod p) using the binary method of exponentiation and the Montgomery multiplication where r = 32. Show the steps and temporary values.}
Consider n=p=29, M=g=10, e=a=23.
\begin{center}
\begin{tabular}{c||c|c|c|c|c|c|c} 
iteration & q & g_{0} & g_{1} & u_{0} & u_{1} & v_{0} & v_{1}\\
\hline
0 & - & 32 & 29 & 1 & 0 & 0 & 1 \\
\hline
1 & 1 & 29 & 3 & 0 & 1 & 1 & -1 \\
\hline
2 & 9 & 3 & 2 & 1 & -9 & -1 & 10\\
\hline
3 & 1 & 2 & 1 & -9 & 10 & 10 & -11\\
\hline
4 & 2 & 1 & 0 & 10 & -29 & -11 & 32 \\
\hline
& & $\bullet$ & & $\bullet$ & & $\bullet$ & \\
\end{tabular}
\label{tab:gcd2}
\end{center}
From Table \ref{tab:gcd2}, GCD = 1, $r^{-1}$ = 10, $n'$ = 11. \\
\begin{itemize}
\item Step 2
\begin{description}
$\overline{M}$ = M * r mod n = 10 * 32 mod 29 = 1
\end{description}
\item Step 3
\begin{description}
$\overline{C}$ = 1 * r mod n = 1 * 32 mod 29 = 3
\end{description}
\item Step 4
\begin{description}
\begin{center}
\begin{tabular}{c|c|c} 
$e_{i}$ & Step 5 & Step 6\\
\hline
1 & MonPro(3,3) = 3 & MonPro(1,3) = 1 \\
\hline
0 & MonPro(1,1) = 10 &  \\
\hline
1 & MonPro(10,10) = 14 & MonPro(1,14) = 24\\
\hline
1 & MonPro(24,24) = 18 & MonPro(1,18) = 6\\
\hline
1 & MonPro(6,6) = 12 & MonPro(1,12) =  4 \\
\end{tabular}
\end{center}
\\
$MonPro(3,3)$\\
t = 3 * 3 = 9\\
m = 9 * 11 mod 32 = 3\\
u = (9 + 3 * 29) / 32 = 3\\
\\
$MonPro(1,3)$\\
t = 1 * 3 = 3\\
m = 3 * 11 mod 32 = 1\\
u = (3 + 1 * 29) / 32 = 1\\
\\
$MonPro(1,1)$\\
t = 1 * 1 = 1\\
m = 1 * 11 mod 32 = 11\\
u = (1 + 11 * 29) / 32 = 10\\
\\
$MonPro(10,10)$\\
t = 10 * 10 = 100\\
m = 100 * 11 mod 32 = 12\\
u = (100 + 12 * 29) / 32 = 14\\
\\
$MonPro(1,14)$\\
t = 1 * 14 = 14\\
m = 14 * 11 mod 32 = 26\\
u = (14 + 26 * 29) / 32 = 24\\
\\
$MonPro(24,24)$\\
t = 24 * 24 = 576\\
m = 576 * 11 mod 32 = 0\\
u = (576 + 0 * 29) / 32 = 18\\
\\
$MonPro(1,18)$\\
t = 1 * 18 = 18\\
m = 18 * 11 mod 32 = 6\\
u = (18 + 6 * 29) / 32 = 6\\
\\
$MonPro(6,6)$\\
t = 6 * 6 = 36\\
m = 36 * 11 mod 32 = 12\\
u = (36 + 12 * 29) / 32 = 12\\
\\
$MonPro(1,12)$\\
t = 1 * 12 = 12\\
m = 12 * 11 mod 32 = 4\\
u = (12 + 4 * 29) / 32 = 4\\
\end{description}
\item Step 7
\begin{description}
C = MonPro(4,1) = 11
\end{description}
\end{itemize}
Result of $10^{23}$ (mod 29) = $\textbf{11}$
\section{Let an RSA key be determined by the parameters \{p,q,n,e,d\} = \{17,23,391,29,85\}. Compute S = $M^{d}$ (mod n) for M = 175 with and without the Chinese remainder theorem and the binary exponentiation.}
\subsection{Chinese remainder theorem}
$d_{1}$ = 85 mod (16) = 5\\
$d_{2}$ = 85 mod (22) = 19\\
\begin{center}
\begin{tabular}{c||c|c|c|c|c|c|c} 
iteration & quotient & g_{0} & g_{1} & u_{0} & u_{1} & v_{0} & v_{1}\\
\hline
0 & - & 23 & 17 & 1 & 0 & 0 & 1 \\
\hline
1 & 1 & 17 & 6 & 0 & 1 & 1 & -1 \\
\hline
2 & 2 & 6 & 5 & 1 & -2 & -1 & 3\\
\hline
3 & 1 & 5 & 1 & -2 & 3 & 3 & -4\\
\hline
4 & 5 & 1 & 0 & 3 & -17 & -4 & 23 \\
\hline
& & $\bullet$ & & $\bullet$ & & $\bullet$ & \\
\label{tab:gcd3}
\end{tabular}
\end{center}
In Table \ref{tab:gcd3}, Initial values of $g_{0}$ = q = 23 and $g_{1}$ = p = 17. This $p^{-1}$ = -4 and $q^{-1}$ = 3.\\
$M_{1}$ = $M^{d_{1}}$ mod p = $175^{5}$ mod 17 = 14.\\
$M_{2}$ = $M^{d_{2}}$ mod q = $175^{19}$ mod 23 = 10.\\
S = $M_{1}$ + p *(($M_{2}$ - $M_{1}$) * $p^{-1}$ mod q) = 14 + 17*((10-14)*-4 mod 23) = 14 + 272 = $\textbf{286}$.
\subsection{Binary Exponentiation.}
Representing 85 in binary results in: 1 0 1 0 1 0 1 \\
Following table shows computation of RSA decryption using binary exponentiation.
\begin{center}
\begin{tabular}{ |c|c|l|l| } 
 \hline
 i & e_{i} & Step 2a & Step 2b \\
 \hline
 \hline 
 6 & 0 & ($(175)^{2}$) mod 391 =  127 & 127 \\ 
 5 & 1 & ($(127)^{2}$) mod 391 = 98 & 98*175 mod 391 = 337 \\ 
 4 & 0 & ($(337)^{2}$) mod 391 =  179 & 179 \\
 3 & 1 & ($(179)^{2}$) mod 391 = 370 & 370*175 mod 391 = 235 \\ 
 1 & 0 & ($(235)^{2}$) mod 391 =  94 & 94 \\ 
 0 & 1 & ($(94)^{2}$) mod 391 = 234 & 234*175 mod 391 = $\textbf{286}$ \\
 \hline
\end{tabular}
\end{center}
\end{document}