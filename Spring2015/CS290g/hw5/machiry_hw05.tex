\documentclass[11pt, pdftex]{article}
\usepackage[margin=1in]{geometry}
\usepackage{graphicx}
\usepackage{amsmath}
\usepackage{listings}
\usepackage{semantic}
\usepackage[hyphens]{url}
\usepackage[breaklinks]{hyperref}
\usepackage[demo]{graphicx}
\usepackage{subcaption}
\title{Homework Assignment 05}
\author{Machiry Aravind Kumar}
\date{UCSB}
\begin{document}
\maketitle
\section{Problem 1}
Let the elliptic curve equation $y^{2} = x^{3} - 3x + 4$ defined over the finite field GF(29) be given.
\subsection{Apply Hasse’s theorem and find the range of the order of the elliptic curve group}
According to Hasse Theorem, we have $p + 1 - 2 \sqrt{p} \leq order \leq p + 1 + 2 \sqrt{p}$. \\
Given P = 29, $\lceil 29 \rceil$ = 6. $p + 1 - 2 \sqrt{p}$ = 18 and $p + 1 + 2 \sqrt{p}$ = 42. \\
Range of the elliptic curve group is: 18 $\leq order \leq$ 42.
\subsection{Compute all elements of the elliptic curve group by enumeration.}

\begin{center}
\begin{tabular}{|l|c|c|c|}
 \hline
x & $x^{3} - 3x + 4$ & y & Points \\
\hline
0 & 4 & $\pm2$ & (0,2), (0,27) \\
\hline
1 & 2 & - &  - \\
\hline
2 & 6 & $\pm8$ & (2,8), (2,21) \\
\hline
3 & 22 & $\pm14$ & (3,14), (3,15) \\
\hline
4 & 27 & - &  - \\
\hline
5 & 27 & - &  - \\
\hline
6 & 28 & $\pm12$ & (6,12), (6,17) \\
\hline
7 & 7 & $\pm6$ & (7,6), (7,23) \\
\hline
8 & 28 & $\pm12$ & (8,12), (8,17) \\
\hline
9 & 10 & - &  - \\
\hline
10 & 17 & - &  - \\
\hline
11 & 26 & - &  - \\
\hline
12 & 14 & - &  - \\
\hline
13 & 16 & $\pm4$ & (13,4), (13,25) \\
\hline
14 & 9 & $\pm3$ & (14,3), (14,26) \\
\hline
15 & 28 & $\pm12$ & (15,12), (15,17) \\
\hline
16 & 21 & - &  - \\
\hline
17 & 23 & $\pm9$ & (17,9), (17,20) \\
\hline
18 & 11 & - &  - \\
\hline
19 & 20 & $\pm7$ & (19,7), (19,22) \\
\hline
20 & 27 & - &  - \\
\hline
21 & 9 & $\pm3$ & (21,3), (21,26) \\
\hline
22 & 1 & $\pm1$ & (22,1), (22,28) \\
\hline
23 & 9 & $\pm3$ & (23,3), (23,26) \\
\hline
24 & 10 & - &  - \\
\hline
25 & 10 & - &  - \\
\hline
26 & 15 & - &  - \\
\hline
27 & 2 & - &  - \\
\hline
28 & 6 & $\pm8$ & (28,8), (28,21) \\
\hline
\end{tabular}
\end{center}
\subsection{Find the exact order of the group.}
Order of the EC group is the number of points on the curve plus one to include point at infinity. From the table above, number of points is equal to 30, thus order of group is {\bf 31}.
\subsection{Find a primitive element of the group. Call that P.}
Since the order of the group is prime, all points except for point at infinity is primitive element. Taking one point: (7, 6). Lets assign: $P=(7,6)$.
\subsection{Compute [15]P using the binary method}
We have $P=(7,6)$.\\
e = 15 = (1111)
\begin{center}
\begin{tabular}{ |c|c|l|l| } 
 \hline
 i & e_{i} & Step 2a & Step 2b \\
 \hline
 \hline 
 2 & 1 & (7, 6) + (7, 6) = (14,26) & (14,26) + (7,6) = (28,21) \\ 
 1 & 1 & (28, 21) + (28, 21) = (2,8) & (2,8) + (7,6) = (19,22) \\ 
 0 & 1 & (19,22) + (19,22) = (13,4) & (13,4) + (7,6) = {\bf (22,28)} \\
 \hline
\end{tabular}
\end{center}
\subsection{Compute [15]P using the canonical recoding binary method}
We have $P=(7,6)$. $-P = (7,-6) = (7,23)$\\
Canonical recording for 15 = f = (1000$\overline{1}$)
\begin{center}
\begin{tabular}{ |c|c|l|l| } 
 \hline
 i & f_{i} & Step 2a & Step 2b \\
 \hline
 \hline 
 3 & 0 & (7, 6) + (7, 6) = (14,26) & (14,26) \\ 
 2 & 0 & (14, 26) + (14, 26) = (17,20) & (17,20) \\ 
 1 & 0 & (17, 20) + (17, 20) = (8,12) & (8,12) \\ 
 0 & $\overline{1}$ & (8,12) + (8,12) = (22,1) & (22,1) + (7,23) = {\bf (22,28)} \\
 \hline
\end{tabular}
\end{center}
\end{document}