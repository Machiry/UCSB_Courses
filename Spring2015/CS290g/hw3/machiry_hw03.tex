\documentclass[11pt, pdftex]{article}
\usepackage[margin=1in]{geometry}
\usepackage{graphicx}
\usepackage{amsmath}
\usepackage{listings}
\usepackage{semantic}
\usepackage[hyphens]{url}
\usepackage[breaklinks]{hyperref}
\usepackage[demo]{graphicx}
\usepackage{subcaption}
\title{Homework Assignment 03}
\author{Machiry Aravind Kumar}
\date{UCSB}
\begin{document}
\maketitle
\section{Problem 1}
Consider the exponent d = 49 = (110001). Show the steps and all intermediate powers in the computation of $m^{d}$ for the algorithms
\subsection{the left-to-right binary method}
\begin{center}
\begin{tabular}{ |c|c|l|l| } 
 \hline
 i & e_{i} & Step 2a & Step 2b \\
 \hline
 \hline  
 4 & 1 & (m)^{2} = m^{2} & m^{2}.m = m^{3} \\ 
 3 & 0 & (m^{3})^{2} = m^{6} & m^{6} \\
 2 & 0 & (m^{6})^{2} = m^{12} & m^{12} \\ 
 1 & 0 & (m^{12})^{2} = m^{24} & m^{24} \\ 
 0 & 1 & (m^{24})^{2} = m^{48} & m^{48}.m = m^{49} \\ 
 \hline
\end{tabular}
\end{center}
\subsection{the right-to-left binary method}
$R_{0}=1,R_{1}=m,i=0$
\begin{center}
\begin{tabular}{ |c|c|c|l|l| } 
 \hline
 i & d_{i} & R_{0} & R_{1} \\
 \hline
 \hline  
 0 & 1 & 1.m & m^{2} \\ 
 1 & 0 & m & (m^{2})^{2} \\
 2 & 0 & m & (m^{4})^{2} \\ 
 3 & 0 & m & (m^{8})^{2} \\ 
 4 & 1 & m.m^{16} & (m^{16})^{2} \\ 
 5 & 1 & m^{17}.m^{32} & (m^{32})^{2} \\
 \hline
\end{tabular}
\end{center}
$R_{0} = m^{49}$
\subsection{the square-and-multiply-always algorithm}
$R_{0}=1,R_{1}=1$
\begin{center}
\begin{tabular}{ |c|c|c|l|l| } 
 \hline
 i & d_{i} & b & R_{0} & R_{b} \\
 \hline
 \hline  
 5 & 1 & 0 & R_{0} = 1^{2} & R_{0} = 1.m \\ 
 4 & 1 & 0 & R_{0} = m^{2} & R_{0} = m^{2}.m \\
 3 & 0 & 1 & R_{0} = (m^{3})^{2} & R_{1} = 1.m \\ 
 2 & 0 & 1 & R_{0} = (m^{6})^{2} & R_{1} = m.m \\ 
 1 & 0 & 1 & R_{0} = (m^{12})^{2} & R_{1} = m^{2}.m \\ 
 0 & 1 & 0 & R_{0} = (m^{24})^{2} & R_{0} = m^{48}.m \\
 \hline
\end{tabular}
\end{center}
$R_{0} = m^{49}$
\subsection{the Montgomery powering ladder}
$R_{0}=1,R_{1}=m$
\begin{center}
\begin{tabular}{ |c|c|c|l|l| } 
 \hline
 i & d_{i} & b & R_{b} & R_{d_{i}} \\
 \hline
 \hline  
 5 & 1 & 0 & R_{0} = 1.m & R_{1} = m^{2} \\ 
 4 & 1 & 0 & R_{0} = m.m^{2} & R_{1} = (m^{2})^2  \\
 3 & 0 & 1 & R_{1} = m^{3}.m^{4} & R_{0} = (m^{3})^2 \\ 
 2 & 0 & 1 & R_{1} = m^{6}.m^{7} & R_{0} = (m^{6})^2 \\ 
 1 & 0 & 1 & R_{1} = m^{12}.m^{13} & R_{0} = (m^{12})^2 \\ 
 0 & 1 & 0 & R_{0} = m^{24}.m^{25} & R_{1} = (m^{25})^2 \\
 \hline
\end{tabular}
\end{center}
$R_{0} = m^{49}$
\subsection{the Atomic square-and-multiply algorithm}
$R_{0}=1,R_{1}=m$
\begin{center}
\begin{tabular}{ |c|c|c|l|l|c| } 
 \hline
 i & d_{i} & b_{before} & R_{b} & R_{0} & b_{after}\\
 \hline
 \hline  
 5 & 1 & 0 & R_{0} = 1 & 1.1 & 1\\ 
 5 & 1 & 1 & R_{1} = m & 1.m & 0 \\ 
 4 & 1 & 0 & R_{0} = m & m.m & 1 \\
 4 & 1 & 1 & R_{1} = m & m^{2}.m & 0 \\
 3 & 0 & 0 & R_{0} = m^{3} & m^{3}.m^{3} & 0 \\
 2 & 0 & 0 & R_{0} = m^{6} & m^{6}.m^{6} & 0 \\
 1 & 0 & 0 & R_{0} = m^{12} & m^{12}.m^{12} & 0 \\
 0 & 1 & 0 & R_{0} = m^{24} & m^{24}.m^{24} & 1 \\
 0 & 1 & 1 & R_{1} = m & m^{48}.m & 0\\
 \hline
\end{tabular}
\end{center}
$R_{0} = m^{49}$
\subsection{the Atomic right-to-left algorithm}
$R_{0}=1,R_{1}=m,b=1,i=0$
\begin{center}
\begin{tabular}{ |c|c|c|l|l| } 
 \hline
 i & d_{i} & b = b $\bigoplus$ d_{i} & R_{b} \\
 \hline
 \hline  
 0 & 1 & 0 & R_{0} = 1.m \\ 
 0 & 1 & 1 & R_{1} = m.m \\
 1 & 0 & 1 & R_{1} = m^{2}.m^{2} \\
 2 & 0 & 1 & R_{1} = m^{4}.m^{4} \\
 3 & 0 & 1 & R_{1} = m^{8}.m^{8} \\
 4 & 1 & 0 & R_{0} = m.m^{16} \\
 4 & 1 & 1 & R_{1} = m^{16}.m^{16} \\
 5 & 1 & 0 & R_{0} = m^{17}.m^{32} \\
 5 & 1 & 1 & R_{1} = m^{32}.m^{49} \\
 \hline
\end{tabular}
\end{center}
$R_{0} = m^{49}$
\section{Let an RSA key be determined by the parameters \{p,q,n,$\phi(n)$,e,d\} = \{97,103,9991,9792,2015,8927\}. Compute S = $M^{d}$ (mod n) for M = 25 using each of these DPA-type countermeasure algorithms by selecting suitable random parameters:}
\subsection{Randomizing m, where e is known}
Picking random r = 17.\\
$m^{*}$ = (17)^{2015}.25 mod (9991) = 7111.\\
$S^{*}$ = (7111)^{8927} mod (9991) = 5681. \\
$r^{-1}$ = 4114.\\
$S$ = 5681.4114 mod (9991) = $\textbf{2685}$.
\subsection{Randomizing m, where e is unknown}
Picking random r = 17.\\
$m^{*}$ = 17.25 mod (9991) = 425.\\
$S^{*}$ = (425)^{8927} mod (9991) = 4289. \\
$r^{-1}$ = 4114.\\
$S$ = 4289.4114^{8927} mod (9991) = $\textbf{2685}$.
\subsection{Randomizing m, using a small r}
Selecting l to be 5. $2^{l}$ = 32.\\
Selecting r to be 17 ( $<$ 32).\\
$m^{*}$ = 25 + 17.9991 = 169872.\\
$N^{*}$ = 32*9991 = 319712.\\
$S^{*}$ = (169872)^{8927} mod (319712) = 52640. \\
$S$ = 52640 mod (9991) = $\textbf{2685}$.
\subsection{Randomizing d, using a small r}
Picking random r = 17.\\
$d^{*}$ = 8927+17*9792 = 175391.\\
$S$ = (25)^{175391} mod (9991) = $\textbf{2685}$. 
\subsection{Randomizing d, where $\phi(n)$ is unknown}
Picking random r = 17.\\
$d^{*}$ = 8927+17*(2015*8927-1) = 305803295.\\
$S$ = (25)^{305803295} mod (9991) = $\textbf{2685}$. 
\subsection{Randomizing d, where e is unknown}
Picking random r = 17.\\
$d^{*}$ = 8927-17 = 8910.\\
$S_{1}^{*}$ = (25)^{8910} mod (9991) = 7017. \\
$S_{2}^{*}$ = (25)^{17} mod (9991) = 9120. \\
$S$ = 7017*9120 mod (9991) = $\textbf{2685}$. 
\subsection{Randomizing n, using small random $r_{1}$ and $r_{2}$}
Picking random $r_{1}$ = 17,$r_{2}$ = 29.\\
$m^{*}$ = 25 + 17*9991 = 169872.\\
$N^{*}$ = 29*9991 = 289739.\\
$S^{*}$ = 169872^{8927} mod (289739) = 22667.\\
$S$ = 22667 mod (9991) = $\textbf{2685}$.
\end{document}